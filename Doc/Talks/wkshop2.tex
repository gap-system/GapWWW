\documentclass[11pt]{article}
\usepackage{lslide}
\usepackage{path}
\addtolength{\textheight}{0.66truein} % A4
\addtolength{\textwidth}{0.27truein} % A4
\vertgroup
\parskip 1ex plus 1ex minus 0.2ex
\def\bs{\begin{slide}}
\def\es{\end{slide}}
\def\bi{\begin{itemize}}
\def\ei{\end{itemize}}
\def\GAP{\textsf{GAP}}
\begin{document}
\title[Farewell]{So Long and Thanks for all the Share Packages}
\author{Steve Linton}
\organization{Division of Computer Science, St.~Andrews}
\titlepage
\begin{slide}
\subsection{Where Now}


You can find out more about \GAP:

\bi
\item from the Web site \path|http://www-gap.dcs.st-and.ac.uk/~gap|
\item from the \GAP\ manual
\item from papers that report work involving \GAP\ 
(incomplete bibliography on the WWW site)
\item from the \texttt{gap-forum} and \texttt{gap-trouble} 
e-mail lists (again see the WWW)
\item the \texttt{gap-workshop} alias will also be maintained for a while
\item from the \GAP\ library (or even kernel) code
\item by experiment
\item we are contemplating another workshop in '99; we  hope that
there will be similar workshops elsewhere
\ei
\es
\bs
\subsection{When you use \GAP}

\bi
\item Please write to the \GAP\ forum and tell us the general
direction you are working in. We like to know, it makes us feel
appreciated, and people may have helpful comments
\item Please remember to cite \GAP\ when you publish
\item If appropriate, please consider ``going the extra mile'' to make
your programs generally usable and contributing them. 

 We have an experimental refereeing system for share packages that provides
``official'' recognition for substantial work. Smaller programs can
simply be deposited in our \texttt{incoming} directory.
\item Have fun, and do good work!
\ei
\es
\bs
\subsection{\GAP\ 3.4.4}

\bi
\item The version you have been using this week will be released very
soon
\item After the release there may be bug fixes released and probably
will be new share packages (watch the forum)
\item Our main efforts go into \GAP\ 4 (more later)
\item A few pointers:
\bi
\item All library and package functions have names beginning with a
capital letter
\item It is safest to always use variable names beginning with a small
letter
\item Especially do not use \texttt{C}, \texttt{E}. \texttt{X} or
\texttt{Z} or things may stop working
\item Variables beginning \texttt{My} are also safe
\item A few useful commands -- \texttt{time}, \texttt{Runtime},
\texttt{Profile}, \texttt{LogTo}, \texttt{PrintTo}, \texttt{AppendTo}
\item From the break loop you can see what is going on and try to fix
it -- \texttt{Backtrace} command
\ei
\ei

\es
\bs
\subsection{The Future for \GAP}

\Underbar{Transfer to St Andrews}

\bi
\item Prof Neub\"user retires at the end of July
\item Martin Sch\"onert left academia last September
\item This year the ``centre'' of \GAP\ development will move to St
Andrews

\begin{quote}
A move from an Aachen-based project with international involvement to
an international project coordinated at St Andrews
\end{quote}

In other words, we hope to persuade you, and others world-wide, to do
a lot of the work. We are trying to set up ways to help, and to
acknowledge the effort involved.
\item A \GAP\ Council has been set up to encourage \GAP\ development,
and keep us in touch with \GAP\ users world-wide.
\ei
\es
\bs
\subsection{\GAP\ 4}

\bi
\item Developed in Aachen (mostly) over the last two-three years
\item Re-engineered kernel
\item New library organisation 
\item Many new features -- system and mathematical
\item Alpha test releases (weekly snapshots of our development
version) available now
\item Beta test release (with some documentation) in May or June
\item First full release later this year
\ei
\es
\bs
\subsection{What's Not Changed}
\bi
\item For interactive use, or simple programming, \GAP\ 4 looks a lot
like \GAP\ 3
\item Some commands have changed name
\item Some things will be done much more efficiently 
\ei

\es
\bs
\subsection{The \GAP\ 4 Kernel}
\bi
\item Martin Sch\"onert, Frank Celler
\item Rebuilt from the ground up
\item More efficient memory manager
\item Easier to extend
\item 64 bit clean
\item Faster function calling
\item Streams 
\bi
\item More sophisticated file handling 
\item More flexible communication with other processes
\ei
\item Save/load workspace
\item \GAP\ compiler
\ei
\es
\bs
\subsection{The \GAP\ 4 Library}
\bi
\item Much extended -- Aachen team, including Thomas Breuer, Bettina
Eick, Alexander Hulpke, Heiko Thei\ss en
\item New permutation group code -- Akos Seress
\item Much better handling of vector spaces and algebras 
\item Speed problems with Sets of Sets largely removed -- immutable
lists and records
\item New polycyclic gropup code, handles infinite pc groups, based on
Deep Thought -- Werner Nickel, Wolfgang Merkwitz
\item Iterators and enumerators
\item Many other goodies
\item New structuring 
\ei
\es
\bs
\subsection{\GAP\ 4 Library Structure}
\bi
\item A very brief introduction
\item Most functions a user calls become Operations
\item An Operation is a place-holder for a whole lot of different
functions -- it's Methods
\item All Methods for an Operation compute the same thing, but may
work for different sets of inputs
\item The system will call the ``best'' Method applicable to the
inputs (method selection)
\item Method selection can depend on ``types'' of inputs, on their
relationships and on facts learned about them since they were created
\item Easy to add new methods that work in specific circumstances
\item Fairly easy to add new types of objects, new Operations, etc.
\ei
\es
\bs
\subsection{Thanks}
\bi
\item Three main speakers
\item Other Speakers
\item Mike and Werner
\item Computer Science technical staff
\item \GAP\ developers
\item all of you
\ei
\es
\end{document}





