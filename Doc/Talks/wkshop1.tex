\documentclass[11pt]{article}
\usepackage{lslide}
\usepackage{path}
\addtolength{\textheight}{0.66truein} % A4
\addtolength{\textwidth}{0.27truein} % A4
\vertgroup
\parskip 1ex plus 1ex minus 0.2ex
\def\bs{\begin{slide}}
\def\es{\end{slide}}
\def\bi{\begin{itemize}}
\def\ei{\end{itemize}}
\def\GAP{\textsf{GAP}}
\begin{document}
\title[Welcome]{Welcome to the GAP Workshop}
\author{Steve Linton}
\organization{Division of Computer Science, St.~Andrews}
\titlepage
\begin{slide}
\Largesize
\subsection{Welcome}
\bi
\item to St Andrews
\item to Computational Algebra and Discrete Mathematics
\item to \GAP
\ei
\es
\bs
\subsection{Welcome to St Andrews}
\bi
\item Three conference locations
\bi
\item Here (Maths lecture theatre C) -- morning lectures
\item The physics foyer (down, over bridge) -- coffee, tea
\item The CS 1st year lab (beyond foyer) -- afternoon lab sessions
\ei
\item Telnet from the lab
\item If you need a local UNIX account, use account \texttt{workshop}
on  \texttt{chrystal} (Pentium Pro w. Linux) password \texttt{gap}
\item Local people have coloured badges
\item Please pay Werner for accommodation
\item Workshop dinner -- Thursday -- please confirm/deny with Werner
\ei
\es
\bs
\subsection{Welcome to Computational Algebra}

\Underbar{Why Compute?}
\bi
\item Exploration -- to form or dismiss conjectures or gain insight
\item Resolve finite problems -- sporadic structures, small cases left
over by general theorems
\item As a research area in its own right -- algorithms, algorithmics
\item As a testbed for Computer Science -- type systems, languages,
database concepts
\ei

\es
\bs
\subsection{Approaching Problems in Computational Algebra}

\bi
\item Charlie Sims, some years ago, divided problems into three
classes
\begin{description}
\item[I] Problems where many group elements fit into memory
\item[II] Problems where just a few group elements fit
\item[III] Problems where only a fraction of one element fits
\end{description}
\item I would generalise this and speak of:
\begin{description}
\item[I] Problems that can be solved quickly by entirely automatic
means
\item[II] Problems that can be solved using standard
low-level tools and some mathematical ingenuity
\item[III] Problems that require special-purpose programming
\end{description}
\item These boundaries are both fuzzy and moving
\item Frustration arises when you tackle a problem in one class using
tools suitable for another
\ei
\es
\bs
\subsection{Welcome to GAP}

\bi
\item Groups, Algorithms, Programming
\item Free system for computational algebra and discrete mathematics
\item Runs under UNIX, MSDOS and MacOS (at least) -- easy to port
\item Developed over 11 years in Lehrstuhl D f\"ur Mathematik, RWTH
Aachen, Germany
\bi
\item Prof Joachim Neub\"user
\item Martin Sch\"onert
\item Cast of thousands, mainly in Aachen, some elsewhere
\ei
\item More info \path|http://www-gap.dcs.st-and.ac.uk/~gap/|
\item If you use \GAP\ in research that you publish, please cite it 
\ei
\es
\bs
\subsection{The parts of \GAP}

\bi
\item C kernel
\bi
\item Critical functions
\item System interface
\item User interface
\item \GAP\ language
\ei
\item \GAP\ library
\bi
\item Most of the real functionality
\item Examples of good \GAP\ programming
\ei
\item Data libraries\\
Character tables and tables of marks, 
various types of group; (nearly) all groups of order up to 1000 
\item Documentation  -- LaTeX files, also browsable as on-line help
and HTML -- about 1600 pages in total
\item Share packages -- user contributed and supported
\bi
\item Pure \GAP\ code  -- eg matrix
\item links to external programs -- eg ANU pq program -- usually UNIX only
\ei
\ei

\es
\bs
\subsection{Capabilities}

Just a small selection:

\bi
\item In the library (and kernel)
\bi
\item Permutation groups 
\item Finite soluble groups 
\item Finitely-presented groups
\item Matrix algebras
\item Character tables
\item Generic finite groups
\item Finite fields, cyclotomic fields, algebraic number fields
\item Polynomials
\ei
\item In share packages
\bi
\item Coxeter groups, Hecke algebra, character tables of Chevalley
groups
\item Matrix groups and algebras
\item Semigroups and near-rings
\item Graphs and finite geometries
\item Crystallographic groups
\item Finitely-presented monoids
\item Infinite polycyclic groups
\item Crossed modules
\ei
\ei
\es





\end{document}





