\title{
Handout for the meeting \\
Computational Group Theory Oberwolfach \\
\bigskip
Groups with exponent six }  

%% preamble

\documentclass[12pt]{article}

%% packages

%% macros

\begin{document}

%% title

\author{M.F. Newman}
\date{\today}
\maketitle

%% pagestyle

%% abstract

\begin{abstract}
This is a report on some computational aspects of work in progress 
on presentations for groups with exponent six. 
It is a part of joint work with George Havas, Alice Niemeyer and Charlie Sims.
\end{abstract}

The motivation for this work is a better understanding of the role 
of sixth powers in
presentations for finitely generated groups with exponent six. 
One hope is that a better understanding may
shed some light on the Burnside question for groups with exponent five.

This is primarily a case study of the use of 
implementations of algorithms to get better insight.
These include: coset enumeration (Havas), Knuth-Bendix string rewriting
(Sims) and soluble quotient calculations (Niemeyer) via
{\sf GAP} (Neub\"user, Sch\"onert {\it et al.}), Magma (Cannon {\it et al.}),
Quotpic (Holt and Rees) or standalones.
Some of the implementations have been improved as a result of this
project.

Recall:  $B(2,6)$ has order $2^{28} 3^{25} \approx 2^{68}$. (M.Hall, 1957)\\
and $R(2,5)$ has order $5^{34} \approx 2^{79}$. (Havas, Wall, Wamsley, 1974)

Question: Is there a set of sixth powers which defines $B(2,6)$
reasonably easily? (or at least proves that $B(2,6)$ is finite 
reasonably easily?)

The vague term `reasonably easily' covers a variety of matters such
as the ease of the proof and the size of the set of sixth powers in
terms of the number of them and the lengths of the underlying words.

How many sixth powers did M. Hall use?

There seems no hope in the foreseeable future that routine use of
coset enumeration or string rewriting will yield an answer to the
first question.

M. Hall's proof was entirely by hand calculations whose primary tool
was Reidemeister-Schreier rewriting.

Coset enumeration can be used to support Hall's proof.

\begin{itemize}
\item
A critical lemma (Lemma 4) in his proof is (verbatim): \\
{\em If $H = \{x,a,b\}$ is of exponent six, and if $x^2 = 1,
a^3 = 1, b^3 = 1, \\ xax = a^{-1}, xbx = b^{-1}$, then $\{a,b\}$ is
of exponent three.} \\
Hall proves that $H$ has order 54 using about 16 sixth powers in about
three pages. He describes this as the most difficult lemma of all.
\item
Coset enumeration does this easily with $(ab)^6,(ab^{-1})^6,(xab)^6,
(x(ab)^3)^6$. (1984) And even without the last sixth power it shows 
finiteness.
\item 
Another key step is (Theorem 2.2): \\
{\em If $G$ is the group $\{a,b,c,d\}$ generated by $a,b,c,d$, and if
each of the subgroups $\{a,b,c\}, \{a,b,d\}, \{a,c,d\}, \{b,c,d\}$ is
of exponent three, then $G$ is finite. If further $G$ is of exponent
six, then $G$ is in fact of exponent three.} \\
Hall proves this in about 7 pages and shows that in the first case
$G$ has order at most $3^{17}$.
\item
Coset enumeration also does this quite easily using 32 cubes to get
that the subgroup generated by $\{a,b,c\}$ has index $3^{10}$. 
\item
The first step in Hall's proof is the simple observation that:
if $F$ is a free group of rank 2, then $F^3$ can be generated by
a finite set of cubes. (At least 28 cubes are needed; Hall's
proof does not give, even implicitly, an upper bound on the
number of cubes.)
\item
Coset enumeration shows that there is a set of 28 cubes which
generates $F^3$.
\end{itemize}

On the basis of the last observation one can show, following Hall's
proof, that $2^{124}$ sixth powers suffice to define $B(2,6)$.
Moreover any proof along the same lines needs lots of sixth powers.

One can use computer implementations to help analyse Hall's proof 
more thoroughly.
For example, in his proof of the lemma above Hall considers the unique subgroup
$A$ of index 2 and its commutator subgroup $A'$. With no sixth powers
$A'$ has a 4-element generating set. It is free on a 4-element generating
set (Quotpic). 
Adding only the relation $(xab)^6$ gives that 
the 3-quotients of $A$ appear to grow unboundedly.
Adding $(ab)^6$ as well gives that $A$ has a
largest nilpotent quotient of order 81, and its kernel has a fairly
large abelian quotient.
Adding $(ab^{-1})^6$ as well gives that $A$ has a largest nilpotent
quotient of order 27 and its kernel has a presentation with 15
generators and 37 relations.
Its largest
abelian quotient is elementary abelian of order $2^6$. 
The commutator subgroup $K'$ can be shown, using Quotpic, to have
a presentation with 0 generators.
Hence $H$ can be proved finite using just 3 sixth powers.
This can be confirmed easily using coset enumeration.

It is not too difficult to show that at least 20 sixth powers are needed
to define $B(2,6)$.
It is seems reasonable to guess that 20 is closer than $2^{124}$.

We have been looking at some subgroups and quotient groups of $B(2,6)$ 
which might play a role in more elaborate finiteness proofs. Considering
the role sixth powers play in presentations for them seems a useful
exercise.

In particular we have been studying
$C(3,3) = \langle a,b \mid a^3,b^3, {\rm exponent\ } 6 \rangle$.
This has order $2^{10} 3^3$.

Coset enumeration and string rewriting independently show that there is a
set of 11 sixth powers 
$\{
(ab)^6,\  
(ab^{-1})^6,\  
(abab^{-1})^6,\  
(aba^{-1}b)^6,\  
(ababa^{-1}b^{-1})^6,\  
$
$ 
(abab^{-1}a^{-1}b)^6,\ 
(abab^{-1}a^{-1}b^{-1})^6,\  
(aba^{-1}ba^{-1}b^{-1})^6,\  
(aba^{-1}b^{-1}ab^{-1})^6
\}$
which define $C(3,3)$. 
These are quite extensive calculations.

We have explored other presentations of the form
$\{a,b \mid a^3,b^3, w_1^6,\dots \! ,w_t^6\}.$

Because enumeration and rewriting are relatively expensive, we have
applied filters to check  a presentation of this kind has a
largest soluble  quotient which has exponent six before doing 
enumerations or rewrites.

The filter has a number of stages and has usually been done via
Quotpic (using some scripts).
\begin{enumerate}
\item
Check that there is no nilpotent quotient of order more than $3^3$.
\item
Get a presentation for the subgroup $N$ generated by cubes.
\item
Check whether its largest abelian quotient has order $2^{10}$.
\item
Check that it has no nilpotent quotient of order greater than $2^{10}$.
\item
Make a list of presentations for all the 1023 subgroups of index 2 in $N$.
(We plan to reduce to one representative of each of 47 conjugacy classes.)
\item
Check whether each of these subgroups has its largest abelian quotient of
order $2^9$.
\end{enumerate}

We have found subsets of the above set of 11 sixth powers which pass the
filter and some which coset enumeration shows define $C(3,3)$.

We know that no set of 4 sixth powers suffices to define $C(3,3)$ (with
$a^3,b^3$).

Note that in $G = \langle a,b \mid a^3,b^3 \rangle$ the subgroup $N$
generated by cubes is free of rank 10.
Let $R$ be the second term of the lower exponent-2 central series of $N$.
Then $G/R$ has order $3^3 2^{65}$ -- and is soluble.
We used the soluble quotient program to get a polycyclic presentation
for $G/R$ and hence computed conjugacy classes of sixth powers and 
normal subgroups generated by sixth powers.

 From this we have been able to find sets of 5 sixth powers, such as 
$\{ a^3, b^3,\ (ab)^6,\ 
(aba^2b)^6,\ (aba^2b^2)^6,\ 
(aba^2bab^2)^6,\  (aba^2b^2ab^2)^6\}$
which pass the filter. So far none of them has been shown to define $C(3,3)$.

\bigskip

\end{document}

