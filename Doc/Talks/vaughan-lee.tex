\documentclass[12pt]{article}
\usepackage{a4,amstex}
%\usepackage{a4,amsmath,amsfonts,latexsym}

\begin{document}

\author{M.R.Vaughan-Lee}
\title{Lie relators in varieties of groups}
\date{June 1997}
\maketitle

\begin{abstract}
Mike Newman and I have used the theory of Lie relators to compute the orders
of the free groups in the variety of Engel-4 groups of exponent 5.
\end{abstract}

\section{Introduction}

Let $\mathcal{V}$ be a variety of groups, and let $F$ be the free group of $%
\mathcal{V}$ with free generators $x_1,x_2,\ldots $. Form the lower central
series of $F$%
\[
\gamma _1\geq \gamma _2\geq \ldots \geq \gamma _i\geq \ldots 
\]
by setting $\gamma _1=F$, and $\gamma _{i+1}=[\gamma _i,F]$ for $%
i=1,2,\ldots $. For $i=1,2,\ldots $ let $L_i=\gamma _i/\gamma _{i+1}$ and
think of $L_i$ as a $\Bbb{Z}$-module. Then let 
\[
L=L_1\oplus L_2\oplus \ldots \oplus L_i\oplus \ldots , 
\]
as a direct sum of $\Bbb{Z}$-modules.

Turn $L$ into a Lie ring: if $a\in L_i$ and $b\in L_j$ then we can write $%
a=g\gamma _{i+1}$, $b=h\gamma _{j+1}$ for some $a\in \gamma _i$, $b\in
\gamma _j$. We set $[a,b]=[g,h]\gamma _{i+j+1}\in L_{i+j}$. As a Lie ring, $%
L $ is generated by $a_1,a_2,\ldots $, where $a_i=x_i\gamma _2$ for $%
i=1,2,\ldots $.

Let $Y$ be the free Lie ring with free generators $y_1,y_2,\ldots $, and let 
\[
\pi :Y\rightarrow L 
\]
be the unique homomorphism which maps $y_i$ to $a_i$ for $i=1,2,\ldots $.
The elements of $\ker \pi $ are called \emph{Lie relators} of $\mathcal{V}$.

If $u\in Y$ lies in the kernel of every homomorphism from $Y$ to $L$, then
we say that $u$ is an \emph{identical Lie relator} of $\mathcal{V}$.
Clearly, identical Lie relators are Lie relators.\bigskip

{\Large Is every Lie relator of $\mathcal{V}$ an identical Lie relator? Or
(equivalently) is the associated Lie ring of a relatively free group of $%
\mathcal{V}$ a relatively free Lie ring?} \newpage

\section{Multilinear Lie relators}

The free Lie algebra $Y$ is graded by multidegree. If $a\in Y$ is a Lie
product of the free generators of $Y$ then we assign a multidegree $%
\underline{w}=\left( w_1,w_2,\ldots \right) $ to $a$ if $a$ has degree $w_i$
in $y_i$ for $i=1,2,\ldots $. If $\underline{w}$ is a multidegree, then we
let $Y_{\underline{w}}$ be the $\Bbb{Z}$-submodule of $Y$ spanned by all
products in the free generators which have multidegree $\underline{w}$. Then 
\[
Y=\bigoplus Y_{\underline{w}}, 
\]
where the direct sum is taken over all possible multidegrees.

An element $a\in Y$ is said to be multilinear if $a\in Y_{\underline{w}}$
for some multidegree $\underline{w}=(w_1,w_2,\ldots )$ such that $w_i$
equals $0$ or $1$ for all $i=1,2,\ldots $. It is well known that multilinear
Lie relators of $\mathcal{V}$ are identical Lie relators. \bigskip

{\Large Let $K_{mul}$ be the fully invariant ideal of $Y$ generated by
multilinear Lie relators of $\mathcal{V}$. 
\[
K/K_{mul}\text{{\ is a torsion group under addition.}} 
\]
} \bigskip More prescisely, suppose that $a$ is a Lie relator of $\mathcal{V}
$, and suppose that $a$ lies in the subalgebra of $Y$ generated by $%
y_1,y_2,\ldots ,y_m$. Suppose further that $a$ has degree $n_i$ in $y_i$ for 
$i=1,2,\ldots ,m$. Then {\Large 
\[
n_1!\cdot n_2!\cdot \ldots \cdot n_m!\cdot a\in K_{mul}. 
\]
} \bigskip

\section{Wall's theory of multilinear relators}

There is a beautiful theory of the multilinear Lie relators of $\mathcal{V}$%
, which is due to Wall. Let $w=w(x_1,x_2,\ldots ,x_m)$ be an element of the
free group on $x_1,x_2,\ldots ,x_m$, and suppose that $w=1$ is an identical
(group) relation in $\mathcal{V}$. We let $R$ be the free associative ring
freely generated by the non-commuting indeterminates $X_1,X_2,\ldots $, and
we let $\widehat{R}$ be the ring of formal power series over $\Bbb{Z}$ in
these indeterminates. We can think of $R$ as being embedded in $\widehat{R}$%
. Then $1+X_i$ is a unit in $\widehat{R}$ for $i=1,2,\ldots $, and the
subgroup of the group of units of $\widehat{R}$ generated by $%
1+X_1,1+X_2,\ldots $ is a free group (with $1+X_1,1+X_2,\ldots $ as free
generators). Let $(e_1,e_2,\ldots ,e_m)$ be an $m$-tuple of non-negative
integers, and let 
\begin{eqnarray*}
Z_1 &=&(1+X_1)(1+X_2)\ldots (1+X_{e_1}), \\
Z_2 &=&(1+X_{e_1+1})\ldots (1+X_{e_1+e_2}), \\
&&\ \ldots \ \  \\
Z_m &=&(1+X_{e_1+\ldots +e_{m-1}+1})\ldots (1+X_{e_1+\ldots +e_m}).
\end{eqnarray*}
Now consider the element 
\[
w(Z_1,Z_2,\ldots ,Z_m)\in \widehat{R}. 
\]
We let $T_w^{(e_1,\ldots ,e_m)}$ be the multilinear component of degree 1 in
each of $X_1,X_2,\ldots ,X_{e_1+\ldots +e_m}$ in the power series expansion
of $w(Z_1,Z_2,\ldots ,Z_m)$. Note that this multilinear element lies in $R$.
Then we define a $\Bbb{Z}$-linear Dynkin bracket operator 
\[
\delta :R\rightarrow Y 
\]
by setting 
\[
\delta (X_iX_j\ldots X_k)=\left\{ 
\begin{array}{l}
\lbrack y_i,y_j,\ldots ,y_k]\text{ if }i=1, \\ 
0\text{ otherwise.}
\end{array}
\right. 
\]
Finally we let 
\[
t_w^{(e_1,\ldots ,e_m)}=\delta (T_w^{(e_1,\ldots ,e_m)}). 
\]
\bigskip {\Large Let $W$ be a basis for the identical group relations of $%
\mathcal{V}$, then 
\[
\{t_w^{(e_1,\ldots ,e_m)}\ |\ w\in W,\ e_1,e_2,\ldots ,e_m\geq 0\} 
\]
is a basis for $K_{mul}$.} \bigskip

\section{The variety $\mathcal{E}_4\cap \mathcal{B}_5$}

The variety $\mathcal{E}_4\cap \mathcal{B}_5$ of Engel-4 groups of exponent
5 is determined by the two group identities $[x_2,x_1,x_1,x_1,x_1]=1$, $%
x_1^5=1$. We apply Wall's theory. If we let $w=[x_2,x_1,x_1,x_1,x_1]$, then 
\[
t_w^{(1,4)}=\sum_{\sigma \in \text{Sym}\{2,3,4,5\}}[y_1,y_{2\sigma
},y_{3\sigma },y_{4\sigma },y_{5\sigma }],
\]
\[
t_w^{(2,4)}=\sum_{\sigma \in \text{Sym}\{3,4,5,6\}}[y_1,y_{3\sigma
},y_2,y_{4\sigma },y_{5\sigma },y_{6\sigma }],
\]
\[
t_w^{(3,4)}=\sum_{\sigma \in \text{Sym}\{4,5,6,7\}}[y_1,y_{4\sigma
},y_2,y_3,y_{5\sigma },y_{6\sigma },y_{7\sigma }].
\]
The group identity $x_1^5$ gives us the Lie relator $5y_1$. We show that 
{\Large the associated Lie rings of free 4-Engel groups of exponent 5 are
free Lie rings in the variety $\mathcal{L}$ of Lie rings determined by these
4 identical Lie relators.} We think of $\mathcal{L}$ as a variety of Lie
algebras over $\Bbb{Z}_5$.

We use an idea of Graham Higman's to show that if $L$ is a Lie ring in the
variety $\mathcal{L}$, and if $x\in L$, then Id$_L(x)$ is nilpotent of class
4. In other words, any product of elements of $L$ is zero if it has degree 5
or more in any one element. This means that in considering Lie relators of $%
\mathcal{E}_4\cap \mathcal{B}_5$, we need only consider relators which have
degree 4 or less in each variable. Recall that if $a$ is a Lie relator of
degree $n_i$ in $y_i$ for $i=1,2,\ldots ,m$. Then 
\[
n_1!\cdot n_2!\cdot \ldots \cdot n_m!\cdot a\in K_{mul}. 
\]
Using the relator $5y_1$, this implies that {\Large all Lie relators in
4-Engel groups of exponent 5 are consequences of multilinear relators.}

So the Lie relators of $\mathcal{E}_4\cap \mathcal{B}_5$ are all
consequences of {\Large 
\[
t_{x^5}^{(n)}\ (n\geq 1),\;t_w^{(m,n)}\ (m,n\geq 1).
\]
} With a little more work we show that the Lie relators in the variety of
4-Engel groups of exponent 5 are all consequences of {\Large 
\[
t_{x^5}^{(n)}\;(1\leq \,n<9),\;t_w^{(m,n)}\;(m,n\geq 1,m+n<9).
\]
} It is straightforward to show that these are all consequences of the
defining relators of $\mathcal{L}$.

Finally we are able to show that the free rank $m$ Lie algebra in $\mathcal{L%
}$ has dimension 
\[
m+\sum_{k=2}^m\binom mk(g_k+c_k), 
\]
where $g_k=(k-1)f_{2k}+(k+1)f_{2k-2}$, and where $c_k=0$ for $k>10$, and $%
c_k $ has the value given in the following table for $2\leq k\leq 10$. 
\[
\begin{tabular}{|c|c|c|c|c|c|c|c|c|}
\hline
$c_2$ & $c_3$ & $c_4$ & $c_5$ & $c_6$ & $c_7$ & $c_8$ & $c_9$ & $c_{10}$ \\ 
\hline
3 & 87 & 595 & 1851 & 2996 & 2562 & 1094 & 224 & 35 \\ \hline
\end{tabular}
\]
(Here $f_k$ is the $k$-th Fibonacci number.)

{\Large Hence the free rank $m$ group of the variety of Engel-4 groups of
exponent 5 has order $5^{m+\sum_{k=2}^m\binom mk(g_k+c_k)}$ .}

\end{document}
